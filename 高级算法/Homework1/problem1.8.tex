\paragraph{1.8}~{}

\begin{algorithm}[H]  
    \caption{求$F(z)$}  
    \begin{algorithmic}[1]
        \Require 整数$z(0 \le z \le n-1)$
        \Ensure 以大于$\frac{1}{2}$正确的概率输出$F(Z)$
        \State 等概率在$[0,n-1]$中生成整数$x$
        \State $y \leftarrow (z-x+n)mod \; n$
        \Return{$(F(x)+F(y)) mod \; m$}
    \end{algorithmic}
\end{algorithm}

对于每个数字被篡改的概率为$\frac{1}{5}$

设$r[i]$表示$F(i)$正确,则
$$p(r[i])=\frac{4}{5}$$
$$p(r[x] \land r[y]) \le \frac{16}{25} \le \frac{1}{2}$$

当运行三次后,返回出现次数较多的结果,此时
$$p(Right) \ge C_3^2(1-\frac{16}{25})\frac{16}{25}^2+C_3^3\frac{16}{25}^3=0.704512$$