\paragraph{1.2}~{}

(1)
属于舍伍德算法。

(2)
每次考虑$S'$是$S$的子集,那么显然$|S'|<|S|<n$,则$\exists b<1:|S'|=bn$。
证毕。

(3)
定义$S_{(i)}$为$S$中阶为$i$的元素,$X_{i,j}$为$S_{(i)}$和$S_{(j)}$比较的次数,显然有$0 \le X_{ij} \le 1$。则算法的比较次数为$\sum_{i=1}^{n} \sum_{j>i} X_{i,j}$。
那么易得算法的平均复杂性为$$E[\sum_{i=1}^{n} \sum_{j>i} X_{i,j}]=\sum_{i=1}^{n} \sum_{j>i} E[X_{i,j}]$$
而$$E[X_{ij}]=p_{ij}\times 1+(1-p_{i,j}) \times 0=p_{ij}$$
所以我们考虑求得$p_{ij}$。
分类讨论如下:
\begin{enumerate}[1)]
\item 当$i<k<j$时,令$A$为随机事件$X_{ij}=1$,$B$为随机事件算法首次选中$S_{(i)~(j)}$中元素作为划分元素。
$B$事件必然发生,因为至少会选择一次$S_{(k)}$作为划分元素。则有
$$p(A|B)=\frac{2}{j-i+1}$$
$$p(A)=\sum_{\alpha}p(A|B_{\alpha})p(B_{\alpha})=\frac{2}{j-i+1}$$
\item 当$k \le i<j$时,令$A$为随机事件$X_{ij}=1$,$B$为随机事件算法首次选中$S_{(k)~(j)}$中元素作为划分元素。
$B$事件必然发生,因为至少会选择一次$S_{(k)}$作为划分元素。则有
$$p(A|B)=\frac{2}{j-k+1}$$
$$p(A)=\sum_{\alpha}p(A|B_{\alpha})p(B_{\alpha})=\frac{2}{j-k+1}$$
\item 当$i<j\le k$时,令$A$为随机事件$X_{ij}=1$,$B$为随机事件算法首次选中$S_{(i)~(k)}$中元素作为划分元素。
$B$事件必然发生,因为至少会选择一次$S_{(k)}$作为划分元素。则有
$$p(A|B)=\frac{2}{k-i+1}$$
$$p(A)=\sum_{\alpha}p(A|B_{\alpha})p(B_{\alpha})=\frac{2}{k-i+1}$$
\end{enumerate}
则有
\begin{align}  
    E[T(n)] &= \sum_{i=1}^{n}\sum_{j>i}E[X_{ij}] \nonumber\\
    &= \sum_{i=1}^{k-1}\sum_{i<j\le k}\frac{2}{k-i+1} +\sum_{j=k+1}^{n}\sum_{k\le i<j}\frac{2}{j-k+1}+\sum_{i=1}^{k-1}\sum_{j=k+1}^{n}\frac{2}{j-i+1} \nonumber\\
    &= 2\sum_{i=1}^{k-1}\frac{k-i}{k-i+1} + 2\sum_{j=k+1}^{n} \frac{j-k}{j-k+1} +\sum_{i=1}^{k-1}\sum_{j=k-i+1}^{n-i} \frac{2}{j+1} \nonumber\\
    &= 2\sum_{i=1}^{k-1} (1-\frac{1}{i+1}) + 2\sum_{j=1}^{n-k} (1-\frac{1}{j+1}) +\sum_{i=1}^{k-1}\sum_{j=k-i+1}^{n-i}\frac{2}{j+1} \nonumber\\
    &= 2n-O(log(k))-O(log(n-k)) + \sum_{i=1}^{k-1}\sum_{j=k-i+1}^{n-i} \frac{2}{j+1} \nonumber
\end{align}  
\begin{align}
    \sum_{i=1}^{k-1}\sum_{j=k-i+1}^{n-i} \frac{2}{j+1} &= 2(\sum_{i=1}{n}\sum_{j=2}^{n-i} \frac{1}{j} - \sum_{i=1}{k} \sum_{j=2}^{k-i} \frac{1}{j} - \sum_{i=1}{n-k} \sum_{j=2}^{n-k-i}\frac{1}{j}) \nonumber\\
    &= O(n\, log(n))-O(k\, log(k))-O((n-k)\, log(n-k)) \nonumber\\
    &= O(n\, log(\frac{n}{n-k}))-O(k\, log(\frac{k}{n-k})) \nonumber\\
    &\approx O(n) \nonumber
\end{align}

综上 $$E(T(n)) = O(n)$$