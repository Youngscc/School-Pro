\paragraph{1.5}~{}
考虑克鲁斯卡尔算法求最小生成树的过程我们可以想到,先将所有的边按照边权排序,然后从小到大依次枚举,如果两个端点不在同一个集合,那么将这个点的两个端点所在的集合相连为一个集合,显然在做完所有的操作之后整张图的所有点会被练到一个集合里。
我们再考虑contraction求最小割的算法,其随机挑选边并压缩的过程实际上与克鲁斯卡尔枚举边并将点集相连的操作是等价的。
如果该算法正确,那么满足生成树上最大边两端的两个集合为最小割之后的两个集合,那么必然需要保证最小生成树上的先枚举的$n-2$较小边均不为最小割中的边。
对比contraction算法,上述满足的情况即等价为算法结束时最小割中无边被收缩。
2.6定理2内容如下:

\quad \quad \emph{设$C$是一个min-cut,其大小为$k$,在contraction求最小割算法结束时,$C$中无边被收缩过的概率大于$\frac{2}{n^2}$}

\textbf{对比该定理的证明过程,我们证明该题目算法的结果中除了最小生成树中最大边外,无边在最小生成树中:}

$C$为最小边,$A_i$表示克鲁斯卡尔生成最小生成树时,第$i$步没有选中$C$边,$1 \le i \le n-2$。

在第一步中选中的边在$C$中的概率最多为$\frac{k}{\frac{kn}{2}}=\frac{2}{n}$,即
$$Pr(A_1) \ge 1-\frac{2}{n}$$
在第二步中,若$A_1$发生,则至少有$\frac{k(n-1)}{2}$条边,选中$C$中边的概率为$\frac{2}{n-1}$,即
$$Pr(A_2|A_1) \ge 1-\frac{2}{n-1}$$
在第$i$步中,若$A_1$至$A_i$发生,则有$n-i+1$个节点,即至少有$\frac{k(n-i+1)}{2}$于是
$$Pr(A_i|\bigcap_{1 \le j < i}A_j) \ge 1-\frac{2}{n-i+1}$$
最后我们有
$$Pr(\bigcap_{1 \le i \le n-2}A_i) \ge \prod_{1 \le i \le n-2}(1-\frac{2}{n-i+1})=\frac{2}{n(n-1)}>\frac{2}{n^2}$$

结合该定理,本题算法中最小生成树上的先枚举的$n-2$较小边均不为最小割中的边的概率大于$\frac{2}{n^2}$。

即该算法正确的概率大于$\frac{2}{n^2}$。

故输出最小割的概率为$\Omega(\frac{1}{n^2})$