\paragraph{2.5}~{}

(1)由于共有$2n$个存储位置的数组作为开放寻址散列表存储$n$个数据项,因此$h(x,i)$每次没有找到空位置的概率不超过$\frac{1}{2}$,
因此探测$a$次的概率等于前$a-1$次未找到空位置,第$a$次找到的概率,即不超过$2^{-a}$。

(2)由第一问可得$Pr(X_i=a) \le 2^{-a}$,将$a=2logn$代入得$Pr(X_i=2logn) \le \frac{1}{n^2}$,又由于对于任意正整数$x$,都有$Pr(X_i=x)>Pr(X_i=x+1)$。
则$Pr(X_i \ge 2logn) \le \frac{1}{n^2}$,得证。

(3)
$$
\begin{aligned}
    Pr(X \ge 2logn) &= Pr(\bigcup_{i=1}^{n} X_i \ge 2logn)\\
    &\le \sum_{i=1}^{n} Pr(X_i \ge 2logn)\\
    &\le n*\frac{1}{n^2}\\
    &= \frac{1}{n}\\
\end{aligned}
$$

(4)
$$
\begin{aligned}
    E(X) &= \sum_{i=1}^{n}P(X=i)\times i \\
    &= Pr(x<2logn)E(X|X<2logn) + Pr(x>2logn)E(X|X\ge 2logn) \\
    &\le 1 \times 2logn + \frac{1}{n} \times n\\ 
    &= 2logn + 1\\
    &= O(logn)
\end{aligned}
$$