\paragraph{2.2}~{}

$\bullet$ 首先证明$X_{ij}$两两独立:

任取$i,j,k,l$且满足$i<j,k<l$,需要证明$Pr(X_{ij}=x_1 \cap X_{kl}=x_2) = Pr(X_{ij}=x_1) \times Pr(X_{kl}=x_2) (x_1,x_2 \in {0,1}) $

当$i,j,k,l$互不相同时,该等式显然成立。

当$i,j,k,l$并不互不相同时,不妨设$i=k$,即证明$Pr(X_{ij}=x_1 \cap X_{il}=x_2) = Pr(X_{ij}=x_1) \times Pr(X_{il}=x_2) (x_1,x_2 \in {0,1})$

显然有$$(\forall i,j): Pr(X_{ij}=0) = Pr(X_{ij}=1) = \frac{1}{2}$$

枚举$x_1,x_2$的取值有:
$$
\begin{aligned}
Pr(X_{ij}=0 \cap X_{il}=0) &= Pr(X_i=1 \cap X_j=0 \cap X_l=0) + Pr(X_i=0 \cap X_j=1 \cap X_l=1) \\
 &= \frac{1}{8}+\frac{1}{8} = \frac{1}{4} = Pr(X_{ij}=0) \times Pr(X_{il}=0)\\
Pr(X_{ij}=0 \cap X_{il}=1) &= Pr(X_i=1 \cap X_j=0 \cap X_l=1) + Pr(X_i=0 \cap X_j=1 \cap X_l=0) \\
 &= \frac{1}{8}+\frac{1}{8} = \frac{1}{4} = Pr(X_{ij}=0) \times Pr(X_{il}=1)\\
Pr(X_{ij}=1 \cap X_{il}=0) &= Pr(X_i=1 \cap X_j=1 \cap X_l=0) + Pr(X_i=0 \cap X_j=0 \cap X_l=1)\\
 &= \frac{1}{8}+\frac{1}{8} = \frac{1}{4} = Pr(X_{ij}=1) \times Pr(X_{il}=0)\\
Pr(X_{ij}=1 \cap X_{il}=1) &= Pr(X_i=1 \cap X_j=1 \cap X_l=1) + Pr(X_i=0 \cap X_j=0 \cap X_l=0)\\
 &= \frac{1}{8}+\frac{1}{8} = \frac{1}{4} = Pr(X_{ij}=1) \times Pr(X_{il}=1)\\
\end{aligned}
$$

故$Pr(X_{ij}=x_1 \cap X_{kl}=x_2) = Pr(X_{ij}=x_1 (x_1,x_2 \in {0,1})) \times Pr(X_{kl}=x_2)$,则$X_{ij}$两两独立。

$\bullet$ 然后证明$X_{ij}$不相互独立:

取$1 \le i < j < k \le n$,显然$$Pr(X_{ij}=0 \cap X_{jk}=0 \cap X_{jk}=0) = 0$$
但$$Pr(X_{ij}=0) \times Pr(X_{jk}=0) \times Pr(X_{ik}) = \frac{1}{2} \times \frac{1}{2} \times \frac{1}{2} = \frac{1}{8} \ne 0$$
故$X_{ij}$不相互独立。